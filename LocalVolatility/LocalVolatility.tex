\documentclass{article}
\usepackage[margin=0.5in]{geometry}
\usepackage[UKenglish]{babel}
\usepackage[UKenglish]{isodate}
\usepackage{amsmath,amssymb,amsthm}
\usepackage{bm}%BoldMath
\usepackage{url}
\usepackage{hyperref}

%%%%bra-ket%%%%
\newcommand{\bra}{\langle}
\newcommand{\ket}{\rangle}
%%%%Fig.%%%%
\renewcommand{\figurename}{Fig.}
%%%%large order%%%%
\newcommand{\order}{{\mathcal{O}}}
%%%%partial%%%%
\newcommand{\del}{\partial}
%%%%%%%%%%%%%%%%%%%%%%%%%%%%%%%%%%%%%%%%
%% font                               %%
%%%%%%%%%%%%%%%%%%%%%%%%%%%%%%%%%%%%%%%%
%%%%real pert%%%%
\renewcommand{\Re}{\text{Re}}
%%%%imaginary pert%%%%
\renewcommand{\Im}{\text{Im}}
\newcommand{\nq}{\mathfrak{q}}
\newcommand{\nw}{\mathfrak{w}}


%%%%%%%%%%%%%%%%%%%%%%%%%%%%%%%%%%%%%%%
\title{Local Volatility Model}
\author{Masahiro Ohta}
\date{\today}
%%%%%%%%%%%%%%%%%%%%%%%%%%%%%%%%%%%%%%%


\begin{document}
\maketitle
%\tableofcontents
\bibliographystyle{amsplain}

\section{Setup}\label{sec:Setup}
We assume the following asset process:
\begin{equation}
	dS_t = \sigma(S_t, t, \cdots) S_t dW_t~.
\end{equation}
Here, $\sigma(S_t, t, \cdots)$ is an adapted function that can have other adapted processes as arguments in addition to the asset price $S_t$.
Call option and it defivatives in terms of the strike are
\begin{eqnarray}
	C(K) &=& \BoldMath{E}\left[\left(S_T -K\right)^+\right]~, \\
	\frac{\partial C(K)}{\partial K} &=& -\BoldMath{E}\left[H\left(S_T -K\right)\right]~,\\
	\frac{\partial^2 C(K)}{\partial K^2} &=& \BoldMath{E}\left[\delta\left(S_T -K\right)\right]~.
\end{eqnarray}

\section{Derivations of the Dupire equation}
\subsection{Simple Derivation}
Derman and Kani gave a simple derivation \cite{derman1994riding}.
\subsection{Sophisticated Derivation by using SDE}
This derivation was given by Gatheral \cite{gatheral2011volatility}.
The stochastic difference of call is
\begin{eqnarray}
	dC &=& d\BoldMath{E}\left[ \left(S_T -K\right)^+\right] \\
		 &=&  \BoldMath{E}\left[d\left(S_T -K\right)^+\right].
\end{eqnarray}

\begin{eqnarray}
	f  (S_T) &=& (S_T -K)^+ = (S_T -K) H(S_T - K)~, \\
	f' (S_T) &=& H(S_T - K) + \underbrace{(S_T -K) \delta(S_T -K)}_{=0}~, \\
	f''(S_T) &=& \delta(S_T -K)^+~,
\end{eqnarray}

\begin{eqnarray}
	d\left(S_T -K\right)^+ &=& f'(S_T)dS_T + \frac{1}{2} f''(S_T)dS_T^2 \\
	                       &=& H(S_T-K)\sigma(S_T, T, \cdots)S_T dW_t + \frac{1}{2} \delta (S_T-K) \sigma^2(S_T, T, \cdots)S_T^2 dt.
\end{eqnarray}

\begin{eqnarray}
	\BoldMath{E}\left[d\left(S_T -K\right)^+\right]
	  &=& \frac{1}{2} \BoldMath{E}\left[\delta (S_T-K) \sigma^2(S_T, T, \cdots)S_T^2 \right] dT \\
		&=& \frac{1}{2} \BoldMath{E}\left[\delta (S_T-K)\right] \cdot \BoldMath{E}\left[\sigma^2(S_T, T, \cdots)| S_T = K\right] K^2dT.
\end{eqnarray}
We get
\begin{equation}
	\frac{dC}{dT} = \frac{1}{2} \frac{\partial^2 C}{\partial K^2}\BoldMath{E}\left[\sigma^2(S_T, T, \cdots)| S_T = K\right]K^2.
\end{equation}

\subsection{Using Fokker-Planck equation}

Using Fokker-Planck equation.

\bibliography{LocalVolatility.bib}
\end{document}
